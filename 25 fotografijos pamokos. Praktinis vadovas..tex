\documentclass[a4paper]{book}

\usepackage[utf8]{inputenc}
\usepackage[L7x]{fontenc}
\usepackage[lithuanian]{babel}
\usepackage[svgnames]{xcolor} % Required to specify font color
\usepackage{textcomp}
\usepackage{hyperref}
\usepackage{fullpage}
\usepackage[shortlabels]{enumitem}

\hypersetup{colorlinks, citecolor=black, filecolor=black, linkcolor=black, urlcolor=black}

\title{25 fotografijos pamokos}
\author{V. P. Mikulinas}
\date{1958}

\begin{document}
	\pagenumbering{gobble}% Remove page numbers (and reset to 1)
	\clearpage
	\thispagestyle{empty}

	\maketitle

	\pagebreak
	\clearpage
	\pagenumbering{arabic}% Arabic page numbers (and reset to 1)

	\setcounter{section}{0}
	\section*{Autoriaus žodis}
		Fotografija labai paplito moksle, technikoje, visuomeniniame gyvenime, buityje. Fototgrafija --- tarybinės spaudos pagalbininkas; į laikraščius ir žurnalus dedamos nuotraukos supažindina skaitytojus su Tėvynės gyvenimu, rodo mūsų liaudies darbą, kultūrą, poilsį, nušviečia užsienio gyvenimą. Fotografijų mėgėjų skaičius mūsų šalyje pasiekė kelis milijonus.

		Mūsų pramonė kasmet pagamina milijoną fotoaparatų. Trys tūkstančiai aparatų per dieną. Tai reiškia, kad per metus nauji šimtai tūkstančių tarybinių žmonių papildys fotografų mėgėjų eiles.

		Daugeliui pradedančių domėtis fotografija vėliau ji praverčia kasdieniniame darbe --- mokslinėje ekspedicijoje, gamyklos arba instituto laboratorijoje, įmonės arba kolūkio klube ir t. t.. Kai kurie iš jų, gal būt, virs kvalifikuotais fotografais specialistais. Kai kuriems fotografavimas taps patraukliu kultūringu poilsiuarba saviveikliniu menu, kuris artimas dailininkų kūrybai.

		Fotografijos mėgėjai neperšoks per pradinas jos technikos pakopas. Padėti jiems, kiek tai įmanoma, ir ne tik pačioje pradžioje, --- šios knygos uždavinys.

		``25 fotografijos pamokos''  --- ne vadovėlis, o praktiškas vadovas savarankiškai užsiiminėjantiems nespalvotąja fotografija.

		Pirmojoje knygos dalyje išdėstyta tai, kas labiausiai reikalinga pirmai pažinčiai su fotografija --- nuo to momento, kai pradedantis fotografas pirmąkart ima į rankas aparatą, iki gatavo atspaudo padarymo.

		Antrojoje dalyje detalizuojamos pagrindinės fotografinio proceso stadijos. Ši dalis skirta skaitytojams, jau pažįstantiems fotografijos abėcėlę, mokantiems fotografuoti, ryškinti filmą, spausdinti nuotraukas ir norintiems smulkiau studijuoti fotografavimo techniką.

		Trečiojoje dalyje papasakota, kuriuo būdu galima geriausiai nufotografuoti įvairius objektus. Čia išdėstytas kolektyvinis mūsų šalies ir užsienio fotografų patyrimas. Ši dalis skiriama techniškai pasiruošusiems fotografams mėgėjams.

		Kiekvieną ``pamoką'' nebūtina išmokti vienu prisėdimu: ją galima nagrinėti ir savaitę --- kaip kam išeina.

		Suprantama, kad, norint tapti geru fotografu, negana perskaityti knygą. Ji gali duoti pagrindą savarankiškam darbui, išmokyti taisyklingų veiksmų, apsaugoti nuo klaidų, sužadinti norą tobulintis. Visa kita priklauso nuo fotografo mėgėjo atkaklumo ir daugiausia nuo praktikos.

		\textit{Vertėjo pastaba.} Verčiant knygą į lietuvių kalbą, autorius kai kurias teksto vietas pataisė.
 
	\part{PAGRINDINĖS FOTOGRAFIJOS ŽINIOS}
 	\pagebreak
 
\section{Introduction}
\end{document}