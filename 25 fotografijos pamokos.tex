\documentclass{book}

\usepackage[utf8]{inputenc}
\usepackage[L7x]{fontenc}
\usepackage[lithuanian]{babel}
\usepackage[svgnames]{xcolor} % Required to specify font color
\usepackage{textcomp}
\usepackage{hyperref}
% \usepackage{fullpage}
\usepackage[shortlabels]{enumitem}
\usepackage{graphicx}
\usepackage{wrapfig}
\usepackage{caption}
\usepackage{nicefrac}

\hypersetup{colorlinks, citecolor=black, filecolor=black, linkcolor=black, urlcolor=black}

\title{25 fotografijos pamokos}
\author{V. P. Mikulinas}
\date{1958}

\renewcommand{\thefigure}{\arabic{figure}}
\addto\captionslithuanian{\renewcommand{\chaptername}{Pamoka}}
\addto\captionslithuanian{\renewcommand{\figurename}{pieš.}}
\captionsetup[figure]{labelformat=simple, labelsep=space}

\graphicspath{ {/} }

\begin{document}
	\pagenumbering{gobble}% Remove page numbers (and reset to 1)
	\clearpage
	\thispagestyle{empty}

	\maketitle

	\pagebreak
	\cleardoublepage
	\pagenumbering{arabic}% Arabic page numbers (and reset to 1)
	\chapter*{}
	\section*{Autoriaus žodis}
		Fotografija labai paplito moksle, technikoje, visuomeniniame gyvenime, buityje. Fototgrafija --- tarybinės spaudos pagalbininkas; į laikraščius ir žurnalus dedamos nuotraukos supažindina skaitytojus su Tėvynės gyvenimu, rodo mūsų liaudies darbą, kultūrą, poilsį, nušviečia užsienio gyvenimą. Fotografijų mėgėjų skaičius mūsų šalyje pasiekė kelis milijonus.

		Mūsų pramonė kasmet pagamina milijoną fotoaparatų. Trys tūkstančiai aparatų per dieną. Tai reiškia, kad per metus nauji šimtai tūkstančių tarybinių žmonių papildys fotografų mėgėjų eiles.

		Daugeliui pradedančių domėtis fotografija vėliau ji praverčia kasdieniniame darbe --- mokslinėje ekspedicijoje, gamyklos arba instituto laboratorijoje, įmonės arba kolūkio klube ir t. t.. Kai kurie iš jų, gal būt, virs kvalifikuotais fotografais specialistais. Kai kuriems fotografavimas taps patraukliu kultūringu poilsiuarba saviveikliniu menu, kuris artimas dailininkų kūrybai.

		Fotografijos mėgėjai neperšoks per pradinas jos technikos pakopas. Padėti jiems, kiek tai įmanoma, ir ne tik pačioje pradžioje, --- šios knygos uždavinys.

		``25 fotografijos pamokos''  --- ne vadovėlis, o praktiškas vadovas savarankiškai užsiiminėjantiems nespalvotąja fotografija.

		Pirmojoje knygos dalyje išdėstyta tai, kas labiausiai reikalinga pirmai pažinčiai su fotografija --- nuo to momento, kai pradedantis fotografas pirmąkart ima į rankas aparatą, iki gatavo atspaudo padarymo.

		Antrojoje dalyje detalizuojamos pagrindinės fotografinio proceso stadijos. Ši dalis skirta skaitytojams, jau pažįstantiems fotografijos abėcėlę, mokantiems fotografuoti, ryškinti filmą, spausdinti nuotraukas ir norintiems smulkiau studijuoti fotografavimo techniką.

		Trečiojoje dalyje papasakota, kuriuo būdu galima geriausiai nufotografuoti įvairius objektus. Čia išdėstytas kolektyvinis mūsų šalies ir užsienio fotografų patyrimas. Ši dalis skiriama techniškai pasiruošusiems fotografams mėgėjams.

		Kiekvieną ``pamoką'' nebūtina išmokti vienu prisėdimu: ją galima nagrinėti ir savaitę --- kaip kam išeina.

		Suprantama, kad, norint tapti geru fotografu, negana perskaityti knygą. Ji gali duoti pagrindą savarankiškam darbui, išmokyti taisyklingų veiksmų, apsaugoti nuo klaidų, sužadinti norą tobulintis. Visa kita priklauso nuo fotografo mėgėjo atkaklumo ir daugiausia nuo praktikos.

		\textit{Vertėjo pastaba.} Verčiant knygą į lietuvių kalbą, autorius kai kurias teksto vietas pataisė.
	\part{PAGRINDINĖS FOTOGRAFIJOS ŽINIOS}
	\setcounter{section}{0}
	\chapter{Pažintis su fotografija}
	% \section{}
		\textbf{Fotografinio proceso elementai. --- Fotoaparato konstrukcija. --- Medžiagos fotografijai.}
		\section*{Fotografinio proceso elementai}
			Fotografija taip pavadinta, sujungus graikiškus žodžius \textit{photos} (šviesa) ir \textit{grapho} (rašau), ir, išvertus į lietuvių kalbą, reiškia rašymą šviesa, atvaizdų darymą šviesa.

			Šviesos spinduliai, atšokę nuo kokio nors apšviesto daikto ir praėję pro objektyvą, sudaro fotografinės plokštelės (stiklo) arba filmo šviesai jautriame sluoksnyje nematomą \textit{slaptąjį atvaizdą}, kuris po cheminio apdirbimo pavirsta matomu atvaizdu (sudarytu iš atvirkščių tonų) --- \textit{negatyvu}; iš negatyvo daromas ant šviesai jautraus fotografinio popieriaus atspaudas --- \textit{pozityvas}.

			Tokiu būdu, fotografinei nuotraukai padaryti būtini paeiliui trys etapai:
			\begin{enumerate}[1)]
				\item \textit{fotografavimo procesas}, arba fotografavimas (fotografuojamojo daikto atvaizdo ant fotografinės plošktelės arba filmo padarymas fotoaparatu);
				\item \textit{negatyvinis procesas}, arba ryškinimas (cheminis fotografinės plokštelės arba filmo apdirbimas, siekiant paversti slaptąjį atvaizdą matomu --- negatyvu);
				\item \textit{pozityvinis procesas}, arba spausdinimas (galutinio atspaudo ant fotografinio popieriaus padarymas iš negatyvo).
			\end{enumerate}

			\textbf{Fotografavimas.} Norint fotografuoti, reikia turėti prietaisą, kuriuo būtų galima padaryti šviesinį fotografuojamojo daikto atvaizdą ant šviesai jautraus sluoksnio ir kuris kartu apsaugotų šį sluoksnį nuo pašalinės šviesos. Toks prietaisas yra \textit{fotografijos aparatas}. Pagrindinės jo dalys --- šviesos nepraleidžianti \textit{kamera} ir \textit{objektyvas}. Be šių dalių, fotoaparate yra užraktas, reikiamą laiko tarpą atveriąs šviesai kelią į jautrųjį sluoksnį, ir įtaisas keisti atstumui tarp objektyvo ir kameros užpakalinės sienelės. Šis įtaisas leidžia padaryti ryškų atvaizdą daiktų, esančių vienokiu ar kitokiu atstumu nuo aparato; jame yra matinis stiklas arba kitoks prietaisas ryškumui nustatyti.

			Kad būtų vaizdžiau, visą fotografavimo procesą nagrinėsime, taikydami jį plokšteliniams aparatams (``Fotokor'', ``Moskva 3''). Mėgėjai, turintieji filminius fotoaparatus, teperskaito atidžiai (čia ir toliau), kaip veikia plokštelinis aparatas, kaip apdirbama plokštelė, --- tai padės išsiaiškinti procesus, vykstančius fotografimo ir ryškinimo metu. Darbo filminiais aparatais ypatybes vėliau nagrinėsime smulkiai.

			Prieš fotografavimą aparatas išraukiamas, pastatomas ant stovo --- štatyvo (darant momentinę nuotrauką, aparatą galima laikyti rankose), ir objektyvas nukreipiamas į daiktą, kurį numatoma fotografuoti. Paskui atidaromas objektyvas, kuris projektuoja į matinį stiklą sumažintą ir apverstą šviesinį fotografuojamojo daikto vaizdą. Kad šis atvaizdas būtų aiškiau matomas ir kad nekliudytų krintati iš šonų ir iš užpakalio šviesa, kameros užpakalyje yra stogelis. Aparatas pasukamas taip, kad numatytų fotografuoti daiktų atvaizdas tilptų matiniame stikle. Jeigu daiktų atvaizdai dideli ir netelpa, fotografas su aparatu atsitraukia; jeigu jie per maži ir jei norima padaryti didesnį atvaizdą, aparatas priartinamas prie fotografuojamojo daikto.

			Atvaizdas matiniame stikle, tikriausia, bus neryškus, pasklidas. Tada stumiama priekinė aparato dalis į priekį arba atgal (arba sukamas priekinis objektyvo lęšis) tol, kol atvaizdas tampa visiškai ryškus. Tai vadinama \textit{ryškumo nustatymu}.

			Nustačius ryškumą objektyvas uždaromas, išimamas matinis stiklas, ir į jo vietą įstatoma \textit{kasetė} --- tam tikra plokščia šviesos nepraleidžianti dėžutė su ištraukiamu dangteliu. Į kasetę būda įdėta šviesai jautri \textit{plošktelė}. Dabar, atidarius kasetės dangtelį ir objektyvą, į plokštelę projektuosis tas pats atvaizdas, kuris buvo matomas matiniame stikle.

			Fotografuojant ištraukiamas kasetės dangtelis, o paskui paleidžiamas užraktas ir \textit{eksponuojama}, atseit, leidžiama fotografuojamojo daikto šviesiniam atvaizdui tam tikrą apibrėžtą laiką veikti ploštelę, kad jos jautriame sluoksnyje įvyktų pakitimai, po kurių vėliau bus galima gauti pastovų atvaizdą. Paskui kasetės dangtelis įstumiamas, ir kasetė išimama. Tuo fotografavimo procesas baigiamas.

			Tas laiko tarpas, kurį projektuojamas į plokštelę atvaizdas, vadinamas \textit{išlaikymu}. Šis laikas būna labai įvairus --- nuo tūkstantųjų sekundės dalių iki kelių minučių --- ir nustatomas iš pradžių pagal tam tikras lenteles, o paskui, fotografui įpratus, --- iš akies.

			Nufotografavus kasetė kartu su plokštele nunešama į laboratoriją --- tamsų kambarį, apšviestą tam tikra neaktiniška (neveikiančia plokštelės) šviesa. Jeigu kasetė su plokštele būtų nors akimirką atidaryta paprastoje baltoje šviesoje, šviesai jautrus sluoksnis tuojau pat sugestų (nors akis šio pakitimo ir nepastebės). Dėl to reikia rūpestingai saugoti neišryškintas plošteles nuo dirbtinės ar dienos šviesos.\\

			\textbf{Plokštelės ryškinimas.} Atminkite, kad tamsiai raudonoje laboratorijos šviesoje galima apdirbti plokšteles, kurios vazdinamos ``Izoorto'' (ir filmus ``Ortochrom''). Šių plokštelių apdirbimą aprašysime ir toliau.

			Taigi, laboratorijoje, nekenksmingiausioje raudonoje šviesoje, atidaroma kasetė ir išimama plokštelė. Ant jos paviršiaus nematyti jokio atvaizdo: jis kol kas dar nematomas, slaptas, nors fotografuojant, šviesai paveikus, plokštelės fotografiniame sluoksnyje įvyko šiokių tokių pakitimų.

			Kad slaptasis atvaizdas pasidarytų matomas, plokštelė dedama į lėkštą vonelę, į kurią yra įpilta tam tikro tirpalo --- \textit{ryškalo}. Plokštelė vietomis palaipsniuj tamsėja, ant jos pasirodo įvairaus tamsumo juosvai pilkos spalvos vaizdas. Plokštelės šviesai jautrus sluoksnis pakito tose vietose, kurias fotografuojant veikė šviesa. Kuo stipriau veikė šviesa vienas ar kitas plokštelės vietas, tuo labiau jos pakinta ir, vadinasi, tuo labiau patamsėja ryškale. O nuo tamsiųjų fotografuojamojo daikto dalių atšoka į plokštelę maža šviesos, dėl to šios plokštelės vietos  ryškale beveik nepasikeičia, lieka pieniškai gelsvos.

			Slaptojo atvaizdo vertimas matomu vadinamas \textit{ryškinimu}. Ryškinimą reikia baigti, kai visos atvaizdo detalės išryškėja (tam reikia paprastai nuo 4 iki 7 minučių). Jeigu bus ryškinama per ilgai, plokštelė apsitrauks pilkai, apsidengs vualiu.

			Išryškinta plokštelėskaidriai neprasišviečia, yra gelsvo arba rausvo atspalvio ir lieka jautri šviesai. Kad negatyvas pasidarytų visiškai nejautrus šviesai ir skaidrus, kad iš jo vėliau būtų galima spausdinti ant fotografinio popieriaus, jis, paskalautas vandenyje, paneriamas į kitą tirpalą --- \textit{fiksažą}. Iš fiksažo negatyvas išimamas, kai šviesiosios vietos tampa visiškai skaidrios (fiksavimas trunka 10 -- 15 minučių). Po to negatyvas rūpestingai plaunamas vandenyje ir džiovinamas.

			Kaip jau kalbėta, plokštelę ryškinti reikia nekenksmingoje raudonoje šviesoje. Praėjus kelioms minutėms nuo fiksavimo pradžios, laboratorijoje jau galima įžiebti baltą šviesą --- ji plokštelei nebekenkia.

			Ryškinimas ir fiksavimas kartu vadinami \textit{negatyviniu procesu}. Po šio proceso gauname \textit{negatyvą}, kuriame yra fotografuoto daikto atvaizdas, bet su atvirkščiu šviesių ir tamsių vietų išdėstymu: tamsiosios nufotografuoto daikto vietos čia yra šviesios (net ir skaidrios), o šviesiosios daikto vietos čia tamsios (net ir nepermatomos). Negatyvinis atvaizdas yra tarpinis.\\

			\textbf{Atspaudo ant fotografinio popieriaus darymas.} Galutinis fotografavimo tikslas yra padaryti nufotografuoto daikto teisingų tonų atvaizdą. Tuo tikslu išdžiovintas negatyvas uždedamas ant šviesai jautraus fotografinio popieriaus ir apšviečiamas. Šviesa, praėjusi pro negatyvą, veikia fotografinį popieriaus sluoksnį . Kuo tamsesnės atskiros negatyvo vietos, tuo mažiau šviesos jos praleidžia, ir dėl to po tamsiomis negatyvo vietomis šviesa veikia silpnai, po šviesiomis --- stipriau.

			Fotografijos praktikoje naudojami beveik išimtiniai ryškinamieji fotografiniai popieriai, kuriuose atvaizdas išeina nematomas (slaptas), kaip ir plokštelėje fotografavimo metu. Atvaizdą reikia išryškinti, kad jis pasidarytų matomas --- sudarytas iš tonų, atvirkščių negatyvui, ir atitinkąs fotografuotąjį daiktą. Paskui atspaudas fiksuojamas, plaunamas ir džiovinamas. Šie fotografiniai popieriai apdirbami taip pat kaip ir plokštelės negatyviniame procese, tamsiame kambaryje, bet šviesesnėje --- oranžinėje arba šviesiai raudonoje šviesoje.

			Atspaudas ant fotografinio popieriaus vadinamas \textit{pozityvu}, o jo gavimo iš negatyvo operacijos --- \textit{pozityviniu procesu}.

			Atspaudą ant fotografinio popieriaus galima padaryti ne tik aprašytuoju \textit{kontaktiniu} būdu, kurio atspaudas padaromas tokio pat dydžio kaip ir negatyvas, bet ir \textit{projekciniu} spausdinimo būdu, vadinamuoju \textit{fotografiniu didinimu}. Spausdinant šiuo būdu, padidintas negatyvinis atvaizdas projektuojamas tamsiame kambaryje projekciniu žibintu į šviesai jautrų popierių (panašiai, kaip projektuojamas filmas kine į ekraną).

			Trumpai susipažinę, kaip vyksta pagrindiniai fotografinio proceso etapai, pradėsime nagrinėti smulkiau, praktiškai, kiekvieną iš jų atskirai.

			Ši knyga skirta paprastos juodos-baltos fotografijos procesams. Spalvotosios fotografijos ji neliečia. Daugiasluoksnių spalvotųjų fotografibių medžiagų apdirbimas, ypač spalvotųjų pozityvų darymas ant popieriaus, žymiai sudėtingesnis už atitinkamus juodos-baltos fotografijos procesus. Fotografas mėgėjas gali pereiti prie spalvotosios fotografijos tik po to, kai išmoksta paprasto fotografavimo technikos. O pagal principinę schemą abi fotografijos rūšys yra vienodos.
		\section*{Fotoaparato konstrukcija}
			Susipažinsime iš pradžių, kaip sudarytas fotografijos aparatas, kuriuo padaromas šviesinis (optinis) atvaizdas.
			\subsection*{Pagrindinės fotografijos aparato dalys}
				Žiūrint paskirties ir konstrukcijos, fotografijos aparatai turi vienokius ar kitokius prietaisus, skirtus fotografavimo operacijoms suprastinti, palengvinti ir patikslinti, bet sudaryti jie visi pagal vieną principą. Fotografavimo proceso esmė visada lieka ta pati: objektyvas projektuoja kameroje fotografuojamojo daikto optinį atvaizdą, kuris atsispaudžia ant šviesai jautrios plokštelės arba filmo.

				Šių laikų bendros paskirties fotografijos aparatas sudarytas iš šių pagrindinių dalių: 1) kameros (šviesos nepraleidžiančios dėžtuės); 2) objektyvo (prietaiso optiniam atvaizdui sudaryti); 3) užrakto (mechanizmo, kuriuo šviesinis atvaizdas reikiamą laiko tarpą praleidžiamas į plokštelę arba filmą); 4) mechanizmo ryškumui nustatyti; 5) vaizdo ieškiklio (prietaiso fotoaparatui nutaikyti į fotografuojamąjį objektą).

				Būtinas fotoaparato priedas yra kasetė (arba kitoks prietaisas jautriajai fotografinei medžiagai įdėti).
				\subsubsection*{Kamera}
					Kamera, paprastai kalbant, yra šviesos nepraleidžianti dėžutė, kurios vienoje sienelėje įtvirtintas objektyvas, o priešingoje sienelėje įtaisoma šviesai jautri medžiaga (1 pieš.).
					\begin{figure}[h]
						\centering
						\includegraphics[width=0.8\textwidth]{1-pav}
						\caption{Pirmojo pardavinėjimui skirto fotoaparato pjūvis (Dager, 1839 m. Nuo to laiko kameros schema nepasikeitė)}
						\label{fig:1}
					\end{figure}
					Kamera turi apsaugoti fotografinę plokštelę arba filmą nuo bet kokios pašalinės šviesos. Fotoaparatų kameros arba korpusai būna: a) standūs, dėžutės tipo (``Liubitel''); b) standūs kompaktiški su ištraukiamu objektyvu (FED, ``Zorkij'', ``Kijev''); c) su sustumiamomis dumplėmis, siaurėjančiomis (``Fotokor'', ``Moskva'') arba vienodo skerspjūvio (štatyvinės kameros), panašiomis į armonikos dumples.

					Konstruktoriai stengiasi padaryti tokią kamerą, kuri suglausta užimtų kuo mažiau vietos.
				\subsubsection*{Objektyvas}
					Svarbiausia fotoaparato dalis --- objektyvas. Tai yra optinis prietaisas, projektuojąs į plokštelę arba filmą fotografuojamojo daikto šviesinį atvaizdą.
					\begin{wrapfigure}{r}{0.33\textwidth}
						\centering
						\includegraphics[width=0.33\textwidth]{2-pav}
						\caption{Pusiau suklijuoto keturių lęšių anastigmato ``Industar'' konstrukcinė schema}
						\label{fig:2}
					\end{wrapfigure}
					Paprastas surenkamas lęšis (didinamasis stiklas) duoda pasklidą, neryškų atvaizdą. Dėl to objektyvai, kurie dabar naudojami fotografijoje, paprastai yra sudaryti iš kelių (nuo trijų iki aštuonių) lęšių, kurių iškilumas arba įdubimas (kreivumo spinduliai) ir stiklo sudėtis tiksliai apskaičiuoti ir išlaikyti gaminant. Gretimi lęšiai atskiriami oro tarpu arba suklijuojami. Tokie tobuli objektyvai vadinami \textit{anastigmatais}; jie duoda aiškų ir ryškų atvaizdą. Visuose mūsų šalyje gaminamuose fotoaparatuose įstatyti anastigmatai (2 pieš.).

					Objektyvas montuojamas įtvare, atitinkančiame kamerą, kuriai jis skirtas. Aparatų su centriniu užraktu objektyvai įmontuoti įtvare, kuris sujungtas su užraktu; mažojo formato aparatų objektyvai --- dvigubame įtvare su sriegiais, kurių dėka objektyvą galima pastumti (sukant) išilgai optinės ašies ir tuo nustatyti atvaizdo ryškumą; štatyvinių kamerų objektyvų įtvarai vadinami normaliais įtvarais.

					Ant objektyvo įtvaro išgraviruota: vienos ar kitos rūšies objektyvo pavadinimas, jo židinio nuotolis ir santykinė anga (šviesos stiprumas), o kai kada ir gaminusio fabriko markė bei eilės numeris. Ten pat arba ant centrinio užrakto įtvaro padaryta diafragmų skalė, o daugumoje šiuolaikinių objektyvų --- ir atstumų skalė bei ryškumo zonos skalė.

					Židinio nuotolis ir santykinė anga yra pagrindiniai objektyvą apibūdinantieji duomenys.\\
				
					\textbf{Židinio nuotolis.} Židinio nuotoliu (pagrindiniu) vadinamas atstumas tarp objektyvo optinio centro ir plošktelės (arba filmo), kai ryškiai nustatytas labai tolimo daikto atvaizdas. Jeigu objektyvas nustatytas taip, kad labai tolimų daiktų (pavyzdžiui, trobesio ir kt., esančių ne arčiau kaip už 100 \textit{m} nuo aparato) atvaizdas matiniame stikle išeina ryškus (tai vadinama ryškumo nustatymu begalybei), tai atsumas tarp objektyvo diafragmos plokštumos ir matinio stiklo yra lygus to objektyvo židinio nuotoliui\footnote{Tai galioja objektyvams, kurių diafragmos plokštuma eina per optinį centrą. Daugumai objektyvų tas atstumas tik apytikriai bus lygus objektyvo židinio nuotoliui. Teleobjektyvams šios taisyklės taikyti negalima.}. Kiekvieno objektyvo židinio nuotolis --- tai tas mažiausias atstumas nuo jo optinio centro iki plokštelės, kuriuo tegalima gauti ryškų atvaizdą. Fotografuojant arčiau esančius daiktus, atstumą tarp objektyvo ir plokštelės tenka padidinti; norint daiktą nufotografuoti natūralaus didumo (neperžengiant aparato plokštelės dydžio ribų), reikėtų dumples ištempti dvigubu objektyvo židinio nuotolio dydžiu --- reikėtų panaudoti \textit{dvigubo ilgio} dumples. Iš mūsų šalyje masiškai gaminamų fotoaparatų tiktai ``Fotokor'' turi dvigubo ilgio dumples; dėl to kitais aparatais negalima fotografuoti labai arti (arčiau kaip už 1,3 -- 1,5 \textit{m}) esančių daiktų, nepanaudojus papildomų prietaisų.

					Židinio nuotolis išreiškiamas centimetrais (arba milimetrais). Nuo jo dydžio priklauso objektyvo šviesos stiprumas ir ryškiai atvaizduojamos erdvės zona, daiktų atvaizdų mastelis ir, be to, kiekvienai objektyvo konstrukcijai --- didžiausias plokštelės arba filmo formatas, kuriuo galima padaryti iki kraštų ryškų atvaizdą.

					Fotografuojant iš to paties taško, objektyvas su trumpu židinio nuotoliu duoda mažo formato atvaizdą smulkiu masteliu, objektyvas su ilgu židinio nuotoliu duoda didelio formato atvaizdą stambiu masteliu. Atvaizdų mastelis yra tiesiog proporcingas židinių nuotoliams.

					Normaliais židinių nuotoliais laikomi: 9 \texttimes 12 \textit{cm} negatyvui --- 13,5 centimetro; 6 \texttimes 9 \textit{cm} negatyvui --- 11 centimetrų; 6 \texttimes 6 cm negatyvui --- 7,5 centimetro; mažojo formato negatyvui (24 \texttimes 36 mm) --- 5 centimetrai.\\

					\textbf{Santykinė anga (geometrinis šviesos stiprumas).} Objektyvo šviesos stiprumu vadinama jo galia apšviesti vienokiu ar kitokiu stiprumu kameroje esančios fotografinės medžiagos šviesai jautrų sluoksnį. Šviesos stiprumo dydžio reikšmė didelė: kuo didesnis objektyvo šviesos stiprumas, tuo mažesnio išlaikymo (plokštelės arba filmo apšvietimo laiko) reikia fotografuojant.

					Suprantama, kad objektyvas su didele anga praleidžia daugiau šviesos negu objektyvas su maža anga. Tačiau absoliutus objektyvo skersmens dydis dar nieko nenuliame. Iš tikrųjų: jeigu palyginsime objektyvą su langu, pro kurį į tamsią patalpą (kamerą) patenka šviesa, tai nesunkiai įsitikinsime, kad bet kuris daiktas (plokštelė arba fimas) bus apšviestas tuo stipriau, kuo didesnis pats langas ir kuo arčiau yra tas daiktas.

					Vadinasi, objektyvo šviesos stiprumas priklauso nuo dviejų dydžių: nuo angos dydžio ir nuo židinio nuotolio. Objektyvo šviesos stiprumas tuo didesnis, kuo didesnė jo anga ir kuo trumpesnis jo židinio nuotolis.

					Šis ryšys išreiškiamas \textit{santykinės angos} dydžiu, kuris yra santykis tarp pilnos objektyvo veikančiosios angos\footnote{Pilna veikiančiąja anga vadinama didžiausia objektyvo anga (įeinamasis vyzdys), pro kurią praeina šviesos spindulių pluoštas; ji paprastai lygi pirmajam lęšiui arba kiek mažesnė už jį (išimtis --- plačiakampiai objektyvai).} skersmens ir jo pagrindinio židinio nuotolio (žinoma, abu dydžiai imami vienodais ilgio vienetais). Pavyzdžiui, angos skersmuo (2 \textit{cm}) sutinka su židinio nuotoliu (8 \textit{cm}) kaip 2 : 8; po suprastinimo (padalijus santykį iš pirmojo nario dydžio) gaunama 1 : 4 --- tai ir yra santykinės angos skaitinė reikšmė.

					Fotoaparato FED objektyvo pilnos angos skersmuo 14,3 \textit{mm}, židinio nuotolis 50 \textit{mm}, o padaliję iš pirmojo nario dydžio (14,3), gausime 1 : 3,5.

					Santykinė anga žymima santykiu tarp vieneto ir skaičiaus, rodančio, kiek kartų to objektyvo pilnos angos skersmuo mažesnis už jo židinio nuotolį.

					Šiuolaikiniuose aparatuose būna objektyvai su santykinėmis angomis 1 : 1,5; 1 : 2; 1 : 2,8; 1 : 3,5; 1 : 4; 1 : 4,5; 1 : 6,3. Kuo didesnis antrasis santykio narys, tuo mažesnė pati santykinė anga. Tai suprantama: santykinės angos skaitinė reikšmė yra trupmena. O kadangi \nicefrac{1}{4} yra mažiau už \nicefrac{1}{2}, tai ir santykinė anga 1 : 4 yra mažesnė už angą 1 : 2.

					Objektyvo anga --- tai skritulys; kaip žinome iš geometrijos, skritulių plotai sutinka kaip jų skersmenų kvadratai. Vadinasi, vieno objektyvo šviesos stiprumas sutinka su kito šviesos stiprumu kaip atitinkamų santykių angų skersmenų kvadratai. Tačiau yra paprastesnis būdas nustatyti, kiek kartų vieno objektyvo šviesos stiprumas didesnis už antrojo: didesnįjį dviejų santykinių angų vardiklį reikia padalyti iš mažesniojo vardiklio ir gautą dalmenį pakelti kvadratu (padauginti patį iš savęs). Pavyzdys: sulyginamas šviesos stiprumas objektyvų, kurių santykinės angos 1 : 4,5 ir 1 : 1,5.
					\[
						(4,5 : 1,5)^{2} = 3^{2} = 9.
					\]

					Vadinasi, antrojo objektyvo šviesos stiprumas yra 9 kartus didesnis už pirmojo, ir, kai anga visa atidaryta, viendomis fotografavimo sąlygomis antrajam objektyvui reikia 9 kartus mažesnio išlaikymo (suapvalinus, pavyzdžiui, \nicefrac{1}{100} sekundės vietoj \nicefrac{1}{10} sekundės).

					\textbf{Diafragma.} Ant objektyvo (apatinėje centrinio užrakto dalyje arba tiesiog ant įtvaro) yra eilė didėjančių skaičių, pavyzdžiui, tokių: 4,5 --- 5,6 --- 8 --- 11 --- 16 --- 22 --- 32 (``Moskva'') arba 3,5 --- 4,5 --- 6,3 --- 9 --- 12,5 --- 18 (FED), kurių pirmasis visada sutampa su objektyvo santykinės angos vardikliu.

					Atidarę centrinį užraktą ir nustatę esančią prie skaitmenų rodyklę-slankiklį ties mažiausiu skaičiumi, pamatysime, kad objektyvo anga atidaryta visa. Jeigu slankiklį stumsime didesnių skaičių link, objektyvo anga palaipsniui mažės ir ties didžiausiu skaičiumi bus mažiausia. Prietaisas objektyvo angai reguliuoti vadianamas \textit{diafragma}, o skaičių eilė --- tai šio prietaiso skalė.

					Šių laikų objektyvuose naudojama vadinamoji irisinė diafragma; ji sudaryta iš lapelių, esančių tarp objektyvo lęšių (maždaug jo optinio centro plokštumoje) ir sudarančių beveik apskritą angą. Susieidami arba prasiskirdami, lapeliai palaipsniui keičia objektyvo veikiančiosios angos dydį (3 pieš.).

					Skaičiai diafragmų skalėje yra objektyvo faktinių (veikiančiųjų) santykių angų vardikliai įvairioms slankiklio padėtims. Jie apskaičiuojami tuo pačiu principu, kaip ir pilna santykinė anga, bet skaitiklis, visada lygus vienetui, patogumo dėlei nežymimas (4 pieš.).

					Diafragma vadinama ir pati reguliuojamoji anga, žymint jos dydį atitinkamu skaitiniu rodikliu (diafragma 5,6) arba išreiškiant jį žodžiais (didelė diafragma, maža diafragma).
					\begin{figure}[h]
						\centering
						\includegraphics[width=0.8\textwidth]{3-pav}
						\caption{Irisinė diafragma}
						\label{fig:3}
					\end{figure}
					\begin{figure}[h]
						\centering
						\includegraphics[width=0.8\textwidth]{4-pav}
						\caption{Diafragmos kiekvienos angos dydis žymimas skaičiumi, kuris rodo, kiek kartų angos skersmuo telpa objektyvo židinio nuotolyje}
						\label{fig:4}
					\end{figure}
					Pastaruoju atveju turimas galvoje angos dydis, bet ne skaičius, kuriuo ta anga pažymėta skalėje. Didelė diafragma --- tai didelė anga, bet maži skaičiai (1,5 -- 4,5). Maža diafragma --- tai maža anga, bet dideli skaičiai (11 -- 36). Vidutinė diafragma --- tai 5,6 -- 9\footnote{Diafragmos anga visais atvejais laikoma veikiančioji objektyvo anga. Ji lygi šiek tiek padidintam diafragmos vaizdui, matomam pro priekinį lęšį. 2 V. P. Mikulinas}.

					Kad būtų trumpiau, sutarsime toliau (tik šioje knygoje) objektyvo šviesos stiprumo dydį žymėti vien tik pilnosios santykinės angos vardikliu, panašiai kaip žymima diafragmų skalėje.

					Siaurindamas objektyvo praleidžiamų šviesos spindulių pluoštą, diafragmavimas sumažina plokštelės arba filmo apšvietimo laipsnį, ir dėl to jau reikia pailginti išlaikymą fotografuojant. Kuo mažesnė naudojamoji diafragmos anga, tuo ilgesnis turi būti išlaikymas.

					Reikia atminti, kad, pavyzdžiui, diafragmos 4 anga visai ne 2 kartus mažesnė už diafragmos 2 angą, bet 4 kartus, ir dėl to išlaikymo su šia diafragma reikia ne du, bet keturis kartus ilgesnio.
					\begin{figure}[h]
						\centering
						\includegraphics[width=0.8\textwidth]{5-pav}
						\caption{Sumažėjus skritulio skersmeniui 2 kartus, skritulio plotas sumažėja $2^{2}$, atseit, keturis, kartus. Skritulių $A$, $B$ ir $C$ skersmenys sutinka kaip 1 : \nicefrac{1}{2} : \nicefrac{1}{4}, o jų plotai kaip 1 : \nicefrac{1}{4} : \nicefrac{1}{16}}
						\label{fig:5}
					\end{figure}
					Tai paaiškinama tuo, kad diafragmos angos numeruojamos pagal jų skersmenis, o apskritų angų plotai sutinka kaip skersmenų kvadratai. Dėl to, sumažėjus 2 kartus angos skersmeniui, jos plotas sumažėja $2^{2} = 4$ kartus.

					Šį santykį vaizdžiai paaiškina 5 piešinys. Nesunku įsitikinti, kad vidurinio skritulio $B$ skersmuo lygiai du kartus mažesnis už kairiojo skritulio $A$ skersmenį; tuo tarpu jo plotas keturis kartus mažesnis, ir, vadinasi, anga $B$ praleis šviesos 4 kartus mažiau negu anga $A$. Dešiniojo skritulio $C$ skersmuo yra 4 kartus mažesnis už kairiojo skritulio $A$ skersmenį, o jo plotas 16 kartų mažesnis. Tas pats ir su diafragmos angomis.

					Kintamų diafragmų skalė sudaryta taip, kad kiekvienai gretimai diafragmai išlaikymo reikia dukart ilgesnio (jeigu anga sumažinama) arba dukart trumpesnio (jeigu anga padidinama)\footnote{Išimtis daroma tiktai pilnajai angai, kai jos dydis neįeina į standartinę diafragmų eilę.}. Tokiu būdu, sumažinus angą per dvi skalės padalas, išlaikymas paketurgubėja ir t. t.

					Žemiau, 1 lentelėje, parodoma (šiek tiek apvalinti) dafragmų angų ir reikalingų išlaikymų priklausomybė.
					\begin{table}[h]
						\caption{\textbf{Priklausomybė tarp diafragmos ir išlaikymo}}
						\begin{tabular}{c|c|c|c|c|c|c|c|c|c|c|}
							\hline
							Diafragma & $\frac{1,4}{1,5}$ & 2 & 2,5 & 2,8 & 3,5 & 4 & 4,5 & 5,6 & 6,3 & 6 \\ \hline
							Santykinis išlaikymo dydis & 1 & 2 & 3 & 4 & 6 & 8 & 10 & 16 & 20 & 32 \\
							\hline
						\end{tabular}
						\begin{tabular}{c|c|c|c|c|c|c|c|c|c|}
							\hline
							Diafragma & 9 & 11 & 12,5 & 16 & 18 & 22 & 25 & 32 & 36 \\ \hline
							Santykinis išlaikymo dydis & 40 & 64 & 80 & 128 & 160 & 256 & 320 & 512 & 640 \\
							\hline
						\end{tabular}
					\end{table}

					Iš lentelės matyti, kad diafragmai 32 reikia 50 kartų ilgesnio išlaikymo negu diafragmai 4,5 ir maždaug 500 kartų ilgesnio išlaikymo negu pilnai objektyvo angai 1,5. Jeigu objektyvas, kurio šviesos stiprumas 2, diafragmuojamas iki 5,6, tai išlaikymą reikia pailginti 8 kartus (16 : 2).

					Diafragmų (vadinasi, ir išlaikymų) reikšmės yra vienodos bet kuriems objektyvams, nepriklausomai nuo jų konstrukcijos, židinio nuotolio ir pilno šviesos stiprumo. Jeigu dviejuose objektyvuose, iš kurių vieno švieos stiprumas 1,5, kito --- 4,5, nustatysime diafragmą 8, tai fotografuojant abiem atvejais išlaikymo reikia vienodo\footnote{Geometrinio šviesos stiprumo požiūriu.}.

					Kam gi reikalinga išlaikymą ilginanti diafragma? Žinoma, ne tiktai atvaizdui ant plokštelės ar filmo patamsinti, nors objektyvo praleidžiamos šviesos sumažinimas būna naudingas, kai objektas nušviestas skaisčios saulės, negatyvinė medžiaga labai jautri, objektyvo šviesos stiprumas didelis, trumpiausiasis užrakto automatiškai atliekamas išlaikymas per ilgas, ir be diafragmavimo būtų neišviangiamas negatyvo peršvietimas (perlaikymas).

					Pagrindinė diafragmavimo paskirtis yra gauti ryškų atvaizdą, praplečiant vidanamąją ryškiai vaizduojamos erdvės zoną tais atvejais, kai norima viename negatyve nufotografuoti iš karto objektus, esančius įvairiais atstumais nuo aparato (vieni arti, kiti toliau arba labai toli).

					\textit{Apie praskaidrintuosius objektyvus.} Daugumas mūsų šalyje gaminamų fotoobjektyvų yra praskaidrinti, kitaip sakant, juose sumažintas kiekis šviesos, kuri paprastai atšoka ir išsisklaido nuo lęšių paviršių ir dėl to arba nepasiekia plokštelės ar filmo, arba sukuria vualį. Paskaidrinimas padidina atvaizdo kontrastą, maždaug 30\% sumažina išlaikymą. Be to, paskaidrintas objektyvas sumažina aureolių susidarymą, duoda švaresnį atvaizdą, kai fotografuojama prieš šviesą ir fotografuojant objektus su smarkiai atspindinčiais paviršiais (sniegas saulėje, vanduo, stiklas ir pan.). Objektyvai paskaidrėja, aptraukus lęšių poliruotus paviršius, susisiekiančius su oru, mikroskopiškai plonu skaidriu sluoksniu, kuris suteikia jiems violetiškai melsvą atspalvį. Kad būtų galima įsivaizduoti skaidrinančio sluoksnio storį, pažymėsime, kad jis lygus \nicefrac{1}{1000} paprastos filminės juostelės storio.

					Tai ir visos žinios apie objektyvą, reikalingos pradedančiajam fotografui mėgėjui iš pradžių, juo labiau, kad jau gamykloje į kiekvieną fotoaparatą įdedamas trumpiausias objektyvas, ir rinktis jo nebereikia.
\end{document}