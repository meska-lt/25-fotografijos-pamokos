\documentclass{book}

\usepackage[utf8]{inputenc}
\usepackage[L7x]{fontenc}
\usepackage[lithuanian]{babel}
\usepackage[svgnames]{xcolor} % Required to specify font color
\usepackage{textcomp}
\usepackage{hyperref}
% \usepackage{fullpage}
\usepackage[shortlabels]{enumitem}

\hypersetup{colorlinks, citecolor=black, filecolor=black, linkcolor=black, urlcolor=black}

\title{25 fotografijos pamokos}
\author{V. P. Mikulinas}
\date{1958}

\addto\captionslithuanian{\renewcommand{\chaptername}{Pamoka}}

\begin{document}
	\pagenumbering{gobble}% Remove page numbers (and reset to 1)
	\clearpage
	\thispagestyle{empty}

	\maketitle

	\pagebreak
	\cleardoublepage
	\pagenumbering{arabic}% Arabic page numbers (and reset to 1)
	\chapter*{}
	\section*{Autoriaus žodis}
		Fotografija labai paplito moksle, technikoje, visuomeniniame gyvenime, buityje. Fototgrafija --- tarybinės spaudos pagalbininkas; į laikraščius ir žurnalus dedamos nuotraukos supažindina skaitytojus su Tėvynės gyvenimu, rodo mūsų liaudies darbą, kultūrą, poilsį, nušviečia užsienio gyvenimą. Fotografijų mėgėjų skaičius mūsų šalyje pasiekė kelis milijonus.

		Mūsų pramonė kasmet pagamina milijoną fotoaparatų. Trys tūkstančiai aparatų per dieną. Tai reiškia, kad per metus nauji šimtai tūkstančių tarybinių žmonių papildys fotografų mėgėjų eiles.

		Daugeliui pradedančių domėtis fotografija vėliau ji praverčia kasdieniniame darbe --- mokslinėje ekspedicijoje, gamyklos arba instituto laboratorijoje, įmonės arba kolūkio klube ir t. t.. Kai kurie iš jų, gal būt, virs kvalifikuotais fotografais specialistais. Kai kuriems fotografavimas taps patraukliu kultūringu poilsiuarba saviveikliniu menu, kuris artimas dailininkų kūrybai.

		Fotografijos mėgėjai neperšoks per pradinas jos technikos pakopas. Padėti jiems, kiek tai įmanoma, ir ne tik pačioje pradžioje, --- šios knygos uždavinys.

		``25 fotografijos pamokos''  --- ne vadovėlis, o praktiškas vadovas savarankiškai užsiiminėjantiems nespalvotąja fotografija.

		Pirmojoje knygos dalyje išdėstyta tai, kas labiausiai reikalinga pirmai pažinčiai su fotografija --- nuo to momento, kai pradedantis fotografas pirmąkart ima į rankas aparatą, iki gatavo atspaudo padarymo.

		Antrojoje dalyje detalizuojamos pagrindinės fotografinio proceso stadijos. Ši dalis skirta skaitytojams, jau pažįstantiems fotografijos abėcėlę, mokantiems fotografuoti, ryškinti filmą, spausdinti nuotraukas ir norintiems smulkiau studijuoti fotografavimo techniką.

		Trečiojoje dalyje papasakota, kuriuo būdu galima geriausiai nufotografuoti įvairius objektus. Čia išdėstytas kolektyvinis mūsų šalies ir užsienio fotografų patyrimas. Ši dalis skiriama techniškai pasiruošusiems fotografams mėgėjams.

		Kiekvieną ``pamoką'' nebūtina išmokti vienu prisėdimu: ją galima nagrinėti ir savaitę --- kaip kam išeina.

		Suprantama, kad, norint tapti geru fotografu, negana perskaityti knygą. Ji gali duoti pagrindą savarankiškam darbui, išmokyti taisyklingų veiksmų, apsaugoti nuo klaidų, sužadinti norą tobulintis. Visa kita priklauso nuo fotografo mėgėjo atkaklumo ir daugiausia nuo praktikos.

		\textit{Vertėjo pastaba.} Verčiant knygą į lietuvių kalbą, autorius kai kurias teksto vietas pataisė.
	\part{PAGRINDINĖS FOTOGRAFIJOS ŽINIOS}
	\setcounter{section}{0}
	\chapter{Pažintis su fotografija}
	% \section{}
		\textbf{Fotografinio proceso elementai. --- Fotoaparato konstrukcija. --- Medžiagos fotografijai.}
		\section*{Fotografinio proceso elementai}
			Fotografija taip pavadinta, sujungus graikiškus žodžius \textit{photos} (šviesa) ir \textit{grapho} (rašau), ir, išvertus į lietuvių kalbą, reiškia rašymą šviesa, atvaizdų darymą šviesa.

			Šviesos spinduliai, atšokę nuo kokio nors apšviesto daikto ir praėję pro objektyvą, sudaro fotografinės plokštelės (stiklo) arba filmo šviesai jautriame sluoksnyje nematomą \textit{slaptąjį atvaizdą}, kuris po cheminio apdirbimo pavirsta matomu atvaizdu (sudarytu iš atvirkščių tonų) --- \textit{negatyvu}; iš negatyvo daromas ant šviesai jautraus fotografinio popieriaus atspaudas --- \textit{pozityvas}.

			Tokiu būdu, fotografinei nuotraukai padaryti būtini paeiliui trys etapai:
			\begin{enumerate}[1)]
				\item \textit{fotografavimo procesas}, arba fotografavimas (fotografuojamojo daikto atvaizdo ant fotografinės plošktelės arba filmo padarymas fotoaparatu);
				\item \textit{negatyvinis procesas}, arba ryškinimas (cheminis fotografinės plokštelės arba filmo apdirbimas, siekiant paversti slaptąjį atvaizdą matomu --- negatyvu);
				\item \textit{pozityvinis procesas}, arba spausdinimas (galutinio atspaudo ant fotografinio popieriaus padarymas iš negatyvo).
			\end{enumerate}

			\textbf{Fotografavimas.} Norint fotografuoti, reikia turėti prietaisą, kuriuo būtų galima padaryti šviesinį fotografuojamojo daikto atvaizdą ant šviesai jautraus sluoksnio ir kuris kartu apsaugotų šį sluoksnį nuo pašalinės šviesos. Toks prietaisas yra \textit{fotografijos aparatas}. Pagrindinės jo dalys --- šviesos nepraleidžianti \textit{kamera} ir \textit{objektyvas}. Be šių dalių, fotoaparate yra užraktas, reikiamą laiko tarpą atveriąs šviesai kelią į jautrųjį sluoksnį, ir įtaisas keisti atstumui tarp objektyvo ir kameros užpakalinės sienelės. Šis įtaisas leidžia padaryti ryškų atvaizdą daiktų, esančių vienokiu ar kitokiu atstumu nuo aparato; jame yra matinis stiklas arba kitoks prietaisas ryškumui nustatyti.

			Kad būtų vaizdžiau, visą fotografavimo procesą nagrinėsime, taikydami jį plokšteliniams aparatams (``Fotokor'', ``Moskva 3''). Mėgėjai, turintieji filminius fotoaparatus, teperskaito atidžiai (čia ir toliau), kaip veikia plokštelinis aparatas, kaip apdirbama plokštelė, --- tai padės išsiaiškinti procesus, vykstančius fotografimo ir ryškinimo metu. Darbo filminiais aparatais ypatybes vėliau nagrinėsime smulkiai.

			Prieš fotografavimą aparatas išraukiamas, pastatomas ant stovo --- štatyvo (darant momentinę nuotrauką, aparatą galima laikyti rankose), ir objektyvas nukreipiamas į daiktą, kurį numatoma fotografuoti. Paskui atidaromas objektyvas, kuris projektuoja į matinį stiklą sumažintą ir apverstą šviesinį fotografuojamojo daikto vaizdą. Kad šis atvaizdas būtų aiškiau matomas ir kad nekliudytų krintati iš šonų ir iš užpakalio šviesa, kameros užpakalyje yra stogelis. Aparatas pasukamas taip, kad numatytų fotografuoti daiktų atvaizdas tilptų matiniame stikle. Jeigu daiktų atvaizdai dideli ir netelpa, fotografas su aparatu atsitraukia; jeigu jie per maži ir jei norima padaryti didesnį atvaizdą, aparatas priartinamas prie fotografuojamojo daikto.

			Atvaizdas matiniame stikle, tikriausia, bus neryškus, pasklidas. Tada stumiama priekinė aparato dalis į priekį arba atgal (arba sukamas priekinis objektyvo lęšis) tol, kol atvaizdas tampa visiškai ryškus. Tai vadinama \textit{ryškumo nustatymu}.

			Nustačius ryškumą objektyvas uždaromas, išimamas matinis stiklas, ir į jo vietą įstatoma \textit{kasetė} --- tam tikra plokščia šviesos nepraleidžianti dėžutė su ištraukiamu dangteliu. Į kasetę būda įdėta šviesai jautri \textit{plošktelė}. Dabar, atidarius kasetės dangtelį ir objektyvą, į plokštelę projektuosis tas pats atvaizdas, kuris buvo matomas matiniame stikle.

			Fotografuojant ištraukiamas kasetės dangtelis, o paskui paleidžiamas užraktas ir \textit{eksponuojama}, atseit, leidžiama fotografuojamojo daikto šviesiniam atvaizdui tam tikrą apibrėžtą laiką veikti ploštelę, kad jos jautriame sluoksnyje įvyktų pakitimai, po kurių vėliau bus galima gauti pastovų atvaizdą. Paskui kasetės dangtelis įstumiamas, ir kasetė išimama. Tuo fotografavimo procesas baigiamas.

			Tas laiko tarpas, kurį projektuojamas į plokštelę atvaizdas, vadinamas \textit{išlaikymu}. Šis laikas būna labai įvairus --- nuo tūkstantųjų sekundės dalių iki kelių minučių --- ir nustatomas iš pradžių pagal tam tikras lenteles, o paskui, fotografui įpratus, --- iš akies.

			Nufotografavus kasetė kartu su plokštele nunešama į laboratoriją --- tamsų kambarį, apšviestą tam tikra neaktiniška (neveikiančia plokštelės) šviesa. Jeigu kasetė su plokštele būtų nors akimirką atidaryta paprastoje baltoje šviesoje, šviesai jautrus sluoksnis tuojau pat sugestų (nors akis šio pakitimo ir nepastebės). Dėl to reikia rūpestingai saugoti neišryškintas plošteles nuo dirbtinės ar dienos šviesos.

			\textbf{Plokštelės ryškinimas.} Atminkite, kad tamsiai raudonoje laboratorijos šviesoje galima apdirbti plokšteles, kurios vazdinamos ``Izoorto'' (ir filmus ``Ortochrom''). Šių plokštelių apdirbimą aprašysime ir toliau.

			Taigi, laboratorijoje, nekenksmingiausioje raudonoje šviesoje, atidaroma kasetė ir išimama plokštelė. Ant jos paviršiaus nematyti jokio atvaizdo: jis kol kas dar nematomas, slaptas, nors fotografuojant, šviesai paveikus, plokštelės fotografiniame sluoksnyje įvyko šiokių tokių pakitimų.

			Kad slaptasis atvaizdas pasidarytų matomas, plokštelė dedama į lėkštą vonelę, į kurią yra įpilta tam tikro tirpalo --- \textit{ryškalo}. Plokštelė vietomis palaipsniuj tamsėja, ant jos pasirodo įvairaus tamsumo juosvai pilkos spalvos vaizdas. Plokštelės šviesai jautrus sluoksnis pakito tose vietose, kurias fotografuojant veikė šviesa. Kuo stipriau veikė šviesa vienas ar kitas plokštelės vietas, tuo labiau jos pakinta ir, vadinasi, tuo labiau patamsėja ryškale. O nuo tamsiųjų fotografuojamojo daikto dalių atšoka į plokštelę maža šviesos, dėl to šios plokštelės vietos  ryškale beveik nepasikeičia, lieka pieniškai gelsvos.

			Slaptojo atvaizdo vertimas matomu vadinamas \textit{ryškinimu}. Ryškinimą reikia baigti, kai visos atvaizdo detalės išryškėja (tam reikia paprastai nuo 4 iki 7 minučių). Jeigu bus ryškinama per ilgai, plokštelė apsitrauks pilkai, apsidengs vualiu.

			Išryškinta plokštelėskaidriai neprasišviečia, yra gelsvo arba rausvo atspalvio ir lieka jautri šviesai. Kad negatyvas pasidarytų visiškai nejautrus šviesai ir skaidrus, kad iš jo vėliau būtų galima spausdinti ant fotografinio popieriaus, jis, paskalautas vandenyje, paneriamas į kitą tirpalą --- \textit{fiksažą}. Iš fiksažo negatyvas išimamas, kai šviesiosios vietos tampa visiškai skaidrios (fiksavimas trunka 10 -- 15 minučių). Po to negatyvas rūpestingai plaunamas vandenyje ir džiovinamas.

			Kaip jau kalbėta, plokštelę ryškinti reikia nekenksmingoje raudonoje šviesoje. Praėjus kelioms minutėms nuo fiksavimo pradžios, laboratorijoje jau galima įžiebti baltą šviesą --- ji plokštelei nebekenkia.

			Ryškinimas ir fiksavimas kartu vadinami \textit{negatyviniu procesu}. Po šio proceso gauname \textit{negatyvą}, kuriame yra fotografuoto daikto atvaizdas, bet su atvirkščiu šviesių ir tamsių vietų išdėstymu: tamsiosios nufotografuoto daikto vietos čia yra šviesios (net ir skaidrios), o šviesiosios daikto vietos čia tamsios (net ir nepermatomos). Negatyvinis atvaizdas yra tarpinis.

			\textbf{Atspaudo ant fotografinio popieriaus darymas.} Galutinis fotografavimo tikslas yra padaryti nufotografuoto daikto teisingų tonų atvaizdą. Tuo tikslu išdžiovintas negatyvas uždedamas ant šviesai jautraus fotografinio popieriaus ir apšviečiamas. Šviesa, praėjusi pro negatyvą, veikia fotografinį popieriaus sluoksnį . Kuo tamsesnės atskiros negatyvo vietos, tuo mažiau šviesos jos praleidžia, ir dėl to po tamsiomis negatyvo vietomis šviesa veikia silpnai, po šviesiomis --- stipriau.

			Fotografijos praktikoje naudojami beveik išimtiniai ryškinamieji fotografiniai popieriai, kuriuose atvaizdas išeina nematomas (slaptas), kaip ir plokštelėje fotografavimo metu. Atvaizdą reikia išryškinti, kad jis pasidarytų matomas --- sudarytas iš tonų, atvirkščių negatyvui, ir atitinkąs fotografuotąjį daiktą. Paskui atspaudas fiksuojamas, plaunamas ir džiovinamas. Šie fotografiniai popieriai apdirbami taip pat kaip ir plokštelės negatyviniame procese, tamsiame kambaryje, bet šviesesnėje --- oranžinėje arba šviesiai raudonoje šviesoje.

			Atspaudas ant fotografinio popieriaus vadinamas \textit{pozityvu}, o jo gavimo iš negatyvo operacijos --- \textit{pozityviniu procesu}.

			Atspaudą ant fotografinio popieriaus galima padaryti ne tik aprašytuoju \textit{kontaktiniu} būdu, kurio atspaudas padaromas tokio pat dydžio kaip ir negatyvas, bet ir \textit{projekciniu} spausdinimo būdu, vadinamuoju \textit{fotografiniu didinimu}. Spausdinant šiuo būdu, padidintas negatyvinis atvaizdas projektuojamas tamsiame kambaryje projekciniu žibintu į šviesai jautrų popierių (panašiai, kaip projektuojamas filmas kine į ekraną).

			Trumpai susipažinę, kaip vyksta pagrindiniai fotografinio proceso etapai, pradėsime nagrinėti smulkiau, praktiškai, kiekvieną iš jų atskirai.

			Ši knyga skirta paprastos juodos-baltos fotografijos procesams. Spalvotosios fotografijos ji neliečia. Daugiasluoksnių spalvotųjų fotografibių medžiagų apdirbimas, ypač spalvotųjų pozityvų darymas ant popieriaus, žymiai sudėtingesnis už atitinkamus juodos-baltos fotografijos procesus. Fotografas mėgėjas gali pereiti prie spalvotosios fotografijos tik po to, kai išmoksta paprasto fotografavimo technikos. O pagal principinę schemą abi fotografijos rūšys yra vienodos.
		\section*{Fotoaparato konstrukcija}
			Susipažinsime iš pradžių, kaip sudarytas fotografijos aparatas, kuriuo padaromas šviesinis (optinis) atvaizdas.
			\subsection*{Pagrindinės fotografijos aparato dalys}
				Žiūrint paskirties ir konstrukcijos, fotografijos aparatai turi vienokius ar kitokius prietaisus, skirtus fotografavimo operacijoms suprastinti, palengvinti ir patikslinti, bet sudaryti jie visi pagal vieną principą. Fotografavimo proceso esmė visada lieka ta pati: objektyvas projektuoja kameroje fotografuojamojo daikto optinį atvaizdą, kuris atsispaudžia ant šviesai jautrios plokštelės arba filmo.

				Šių laikų bendros paskirties fotografijos aparatas sudarytas iš šių pagrindinių dalių: 1) kameros (šviesos nepraleidžiančios dėžtuės); 2) objektyvo (prietaiso optiniam atvaizdui sudaryti); 3) užrakto (mechanizmo, kuriuo šviesinis atvaizdas reikiamą laiko tarpą praleidžiamas į plokštelę arba filmą); 4) mechanizmo ryškumui nustatyti; 5) vaizdo ieškiklio (prietaiso fotoaparatui nutaikyti į fotografuojamąjį objektą).

				Būtinas fotoaparato priedas yra kasetė (arba kitoks prietaisas jautriajai fotografinei medžiagai įdėti).
				\subsubsection*{Kamera}
\end{document}